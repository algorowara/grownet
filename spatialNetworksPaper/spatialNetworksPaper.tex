\documentclass[aps,pre,reprint,superscriptaddress,amsmath,amssymb]{revtex4-1}
\usepackage{graphicx}
\usepackage{dcolumn}
\usepackage{bm}
\usepackage{hyperref}
\usepackage{natbib}

\begin{document}

\title{Plum pudding models for growing networks: Emergence of small worlds
from spatial embedding}
\author{Alex Gorowara}
\affiliation{Institute for Research in Electronics and Applied Physics, University of Maryland, College Park, Maryland 20742, USA}
\author{Ari Zitin}
\affiliation{Institute for Research in Electronics and Applied Physics, University of Maryland, College Park, Maryland 20742, USA}
\author{Mark Herrera}
\affiliation{Department of Physics and Institute for Research in Electronics and Applied Physics, University of Maryland, College Park, Maryland 20742, USA}
\author{Shane Squires}
\affiliation{Department of Physics and Institute for Research in Electronics and Applied Physics, University of Maryland, College Park, Maryland 20742, USA}
\author{Tom Antonsen}
\affiliation{Department of Physics, Department of Electrical and Computer Engineering, and Institute for Research in Electronics and Applied Physics, University of Maryland, College Park, Maryland 20742, USA}
\author{Michelle Girvan}
\affiliation{Department of Physics, Institute for Physical Science and Technology, and Institute for Research in Electronics and Applied Physics, University of Maryland, College Park, Maryland 20742, USA}
\author{Edward Ott}
\affiliation{Department of Physics, Department of Electrical and Computer Engineering, and Institute for Research in Electronics and Applied Physics, University of Maryland, College Park, Maryland 20742, USA}

\date{\today}

\begin{abstract}
Researchers have studied the spatial embedding of static complex networks,
but networks in nature are inherently dynamical, gaining links and
nodes as they develop over time. We explore the properties of growing
networks, investigating the relationship between statistical network
properties and the space in which they are embedded. In particular,
we develop a broad class of models that share some but not all network
properties. Through these models we characterize how spatial topologies
support the emergence of small-world networks. 
\end{abstract}

\pacs{05.45.-a, 05.65.+b, 89.75.-k, 89.75.Fb ,89.75.Hc} %Nonlinear dynamics and chaos, Self-organized systems, Complex Systems, Structures and organization in complex systems, Networks and genealogical trees

\maketitle

\section{INTRODUCTION}
%in the introduction don't forget to present definitions we use, in particular be clear what we mean by small world property
%mention how our generalizations were chosen for their physical significance rather than the ease with which we can perform comupational and analytic calculations of their properties. 

\section{THE OHO MODEL}
The original Ozik-Hunt-Ott (OHO) model, presented in \cite{ozik2004} considers a network which initially has a clique of $m+1$ nodes on the circumference of a circle. 
At each discrete time step the network is grown according to the following rules: 
(a) a new node is placed on the circumference of the circle between two existing nodes in a interval selected at random from a uniform distribution;
(b) the new node is linked to its $m$ nearest neighbors, where nearest means the smallest number of internode invervals between the nodes;
(c) the steps are repeated in sequence to generate a networks with $N$ nodes.
Since the network is incremented in size by one node each timestep, the network size $N$ can also be used as the system timeparameter.  
It has been shown \cite{ozik2004} that this growth model leads to a small-world network with an exponentially decaying degree distribution. 
The original goal of the OHO model was to explore the effect that geographic locality has on the growth of networks; in this paper we extend this analysis by considering networks growing by geographic attachment preference in more general Euclidean spaces than the circumference of a circle. 

\subsection{The Original Model: Embedding a Network on a Circle}
The original OHO model exhibits the following three properties which indicate that the small-world property is present:
\begin{description}
  \item[Degree Distribution] It was shown in \cite{ozik2004} that for large network size N the degree distribution $H(k,N)$ approaches an asymptotically N invariant form given by 
\[\bar{H}(k) = \frac{1}{m+1}\left(\frac{m}{m+1}\right)^{k-m}\]
for $k \geq m$ and $\bar{H}(k) = 0$ for $k < m$.
This leads to a result that as N approaches infinity, the average node degree $\langle k \rangle$ approaches the limiting value $\langle k \rangle {_\lim_{N \to \infty}} = 2m$.
This means that the average node degree remains finite even as $N \to \infty$, which shows that the OHO model meets the first criterion for the small-world property.
  \item[Characteristic Path Length] Although analytic results for the relationship between the shortest path length between pairs of nodes averaged over the whole network (which we denote $\ell$) is generally impossible, simulation results allow the verification of the desired small world path lenth scaling: $\ell \sim \log N$.
This property was verified for the OHO model in \cite{ozik2004}, but an intuitive picture of why this is the case can be seen as follows, as new nodes are added they push apart the older connected nodes, leaving long edges that pass through the middle of the circle. 
These older nodes then serve as hubs for travel between pairs of nodes across the network, dramatically decreasing the shortest path length between any given pair of nodes.
  \item[Clustering Coefficient] The final criterion for a small world network is comparatively high clustering coefficient, or to be more precise, a clustering coefficient that does not decay to $0$ as $N \to \infty$. 
In order for this criterion to make sense, we must provide a clear definition of this clustering coefficient. 
In their seminal paper, Watts and Strogatz defined the clustering coefficient (which we will denote $C_{ws}$) as the average over the whole network of the clustering coefficient $C_i$ of each node $i$ \cite{wsnat}.
For each node the local clustering $C_i$ is defined as $C_i = \frac{q_i}{\frac{1}{2} k_i(k_i-1)}$ where $q_i$ is the number of links between the $k_i$ neighbors of node $i$ and $\frac{1}{2} k_i(k_i-1)$ is the maximum possible number of such links between the neighbors of node $i$.
This leads to a natural expression for the clustering coefficient, $C_{ws} = \frac{1}{N} \sum_{i=1}^{N} C_i$, which is shown in \cite{ozik2004} to approach an asymptotic value $C_{ws} = 0.648$ as $N \to \infty$.
As a note to the reader, in the past few years new measures of clustering have been developed that do a better job representing the global clustering in the network, and although for consistency with \cite{ozik2004} we used the Watts-Strogatz definition of clustering in the preceding discussion, we will later use some different definitions that will be presented as they arise.
\end{description}
Thus the OHO model produces a small-world network, but we wonder what general features of this model leads to the small-world property, and so we investigate similar models to explore how the small-world property emerges from spatially embedded networks.

\subsection{Generalizing the Model: The Thomson Problem}
The natural generalization from embedding nodes on the one-dimensional circumference of a circle is to embed them on the two-dimensional surface of a sphere, or in general on the n-dimnsional surface of a hypersphere.
On a circle it is trivial to arrange N points equidistant along the circumference, but the analogous procedure is much more difficult on higher dimensional surfaces. 
The idea of finding a configuration of points on the surface of a sphere such that each point is equidistant to all of its neighbors dates back to 1904 when J.J. Thomson introduced his plum pudding model of the atom, a key problem for him was arranging the electrons (corpuscles) on the surface of the sphere of pudding such that the potential energy is minimized \cite{thomson1904}.

We take this idea and model the nodes as point charges confined to exist on the surface of a sphere of unit radius; and each time step we require that the potential energy between the nodes is minimized.
In this way we produce a configuration of nodes on the sphere's surface such that each internode area interval is approximately (up to numerical defects) equal, implying that if we were to select a point on the sphere uniformly at random, each area interval would be equally likely to be selected.
Now at each timestep we drop a node into an area interval selected at random on the sphere and add links to connect it to its $m$ nearest neighbors, where here distance is defined as the shortest great circle path along the surface of the sphere between two nodes.
In this way we reproduce the key features of the OHO model, a growing network with new nodes only making local connections, but we also introduce a new parameter, the dimension of the space in which the growing network is embedded.
In the following sections we explore the differences between the OHO model and the higher dimension generalizations, starting with the sphere case. 

\subsubsection{The 2D Case: Network Embedded on the Surface of a Sphere}
The first generalization we explore is the case of nodes confined to exist on the surface of a unit sphere in Euclidean 3-space.
This is directly analogous to the original Thomson problem presented in \cite{thomson1904}, where N electrons are bound to the surface of a sphere and interact with each other via the Couloumb force.
As a result of the Couloumb interaction, the minimum energy configuration for these points on the sphere is one where the points are distributed in an approximately uniform manner across the surface of the sphere.
In order to determine if this model produces a small world network we determine the following three properties:
\begin{description}
  \item[Degree Distribution] The uniform distribution of nodes on the surface of the sphere leads to a uniform probability of choosing an area interval in which to place a new nodes; thus existing nodes each have an equal probability of having their degree increased at each timestep.
This causes the degree distribution for this new model to be identical to the degree distribution for the original OHO model, for more detail on this discussion refer to the section on invariant properties at the end of this article.
  \item[Characteristic Path Length] For this model we find that the the average shortest path length $\ell$ scales logarithmically with the network size N, that is, $\ell \sim \log N$. 
This is a reasonable result because as the network grows in size the older nodes get pushed apart by the repulsive Coulumb force thus leaving bridges across the network that span a significant physical distance.
These long range links serve to connect disparate regions of highly interconnected nodes, dramatically reducing the shortest path length between any two nodes in the network.
At each timestep only geographically local connections are made, but due to the dynamic nature of the nodes' spatial positions, each timestep can make existing links longer in physical space, thus building bridges across the network.
Furthermore, we find that for a given value of $N$ the shortest path between any two nodes in this 2D case is shorter than that of the corresponding 1D case (the original OHO model).
One possible explanation for this feature is that the surface of a sphere provides more freedom for nodes to move around each other, thus increasing the chance that a shortcut is created.
In the original OHO model each node, like a boson in a Tonks-Girardeau gas, is forever locked between its two original spatial neighbors, and thus long range links can only be created if new nodes are placed between the two neighbors.
In this first generalization of the OHO model, the repulsive Couloumb force between electrons allows them to rearrange thier relative positions by moving in 2 dimensions in order to minimize the potential energy of the configuration.
Thus two originally adjacent nodes can be moved apart around other nodes, forming long range links; of course placing new nodes between them will also create bridges, but these effects combine to produce shorter path length in the 2D case than in the 1D case.
  \item[Clustering Coefficient] 
%I'm going to leave this section blank until we have results and figures for this since that should determine what we write.
\end{description}
Thus we find that that the model of a network growing on the surface of a sphere with geographic attachment preference leads to the emergence of the small-world property.
We now move onto discussing the generalized case of a network embedded on the surface of an N-sphere in order to see how the dimension of the embedding space impacts the network properties.

\subsubsection{Spatial Embedding of Networks on the Surface of the Unit N-Sphere}
%Alex should probably write this section

\section{THE PLUM PUDDING MODEL}
A key feature of the OHO model is that the nodes can only be placed on the circumference of the circle, when we generalize this to higher dimensions we are still embedding the network on an $n$-dimensional surface in an $n+1$ dimensional space, so nodes never occupy a volume in the embedding space.
In this generalization we explore how the network properties are affected if we instead embed a growing network in an n-dimensional without restricting them to a spherical shell.  In order to ensure unbiased behavior over all nodes, we instead confine them to an $n$-dimensional ball. 

Keeping with the idea of nodes as point charges that we introduced earlier, we model our network as a collection of classical electrons, each electron representing a node, free to move around in a "cloud" of positive charge (which for our model fills the embedding space).
This model was motivated by Thomson's defunct atomic model in which electrons were plums embedding in a sphere of positively charged pudding.
We use this model because the repulsive potential between electrons ensures that the nodes are spaced equidistant from one another in the minimum potential configuration, while the cloud of positive charge ensures that the nodes do not move arbitrarily far from some fixed starting point.
Due to the nature of Gauss' Law in different dimensions we find that for each dimension the potential between any given node and the cloud of positive charge is proportional to the square of the distance between the center of the embedding volume and the node in question.
This means that we can visualize this model as a collection of classical electrons all connected to some origin point by springs (a harmonic potential) and interacting with each other via the dimensionally appropriate Couloumb force.  In order to maintain the unit $n$-ball, this potential is kept proportional to the number of nodes in the graph, and in the Plum Pudding analogy can be said to have a "charge" equal and opposite to the sum of "charges" over all of its nodes.
Like the OHO model we place new nodes randomly in the volume and connect them to their $m$ nearest neighbors, where here we define nearest to be the Euclidean distance between the nodes.
After each node is placed, the configuration is rearranged in order to minimize the total potential energy of the system, which forces the nodes to arrange themselves in an approximately uniform distribution throughout the volume in the embedding space.

%should we include a picture of the 2D Plum Pudding with age colored? It's a good picture, but I'm just not sure how relevant it is.


\subsection{Boundary Conditions on the OHO Model}
First we consider the low-dimensional cases and compare them to the original OHO model and its generalizations to see how these models differ.
The key difference is that when we embed networks on a smooth surface like in the previous models there are no hard boundaries beyond which the nodes cannot move.
In the plum pudding model on the other hand, if nodes move too far from the origin (the center of the volume in which the nodes are embedded) they hit a potential boundary beyond which they cannot pass.
These boundary conditions lead to differences in network properties such as the clustering coefficient between the two models (plum pudding model and the generalized OHO model).

\subsubsection{The 1D Case: Nodes on a Line}
The laws of electromagnetism in 1 dimension lead to a constant force term between each electron, and thus this case is really just a line segmented into equal parts by the set of nodes. 
This model is essentially the same as the original OHO model (nodes on the circumference of a circle), with the exception of a boundary condition. 
On the circle each node borders two internode intervals while on the line the nodes at each endpoint only border one internode interval.
Thus this model leads to nearly identical results as the OHO model, especially for large $N$.
For large $N$ there are still only two boundary nodes, but they make up a fraction $\frac{2}{N}$ of all the nodes, and thus barely contribute to the stastical network properties.
For very small values of $N$ we see some minor differences between the two model, but for any value of $N$ large enough to calculate statistics we find agreement between the 1D Plum Pudding model and the original OHO model (which is the 1D case of the generalized thomson model discussed earlier).

%What properties should we include here? We have to put some plots up, maybe comparing 1D PP and TP?


\subsubsection{The 2D Case: Nodes on a Disk}
Although the disk is a 2 dimensional object and the sphere is fundamentally a 3 dimensional object, nodes embedded on the surface of a sphere and nodes embedded on a disk both live in an effectively 2 dimensional space.
As a result we would expect that the properties explored for the case of nodes embedded on the surface of a sphere hold in this case as well.
Since the minimum energy configuration for the nodes in a cloud of positive charge is a uniform distribution across the face of the unit circle, we expect to find an identical degree distribution to that of the prior models.
This is verified and discussed in more depth in the section on invariant properties at the end of this article.
%include some stuff about how clustering and path length look different from sphere if we don't choose the appropriate force law. Perhaps include more about how the force law impacts the properties? 

\subsection{Spatial Embedding of Networks in a unit N-Ball}
%Alex should probably write this section

\section{INVARIANT PROPERTIES}
The preceding sections illustrate how the generalizations of the OHO model lead to some distinct network properties, but our generalizations do not change everything about the networks.
In this section we discuss some properties that are invariant between our different models and explore why the dimension of the space in which the network is embedded does not impact these particular properties.
One key feature that leads to the invariance discussed below is the uniform probability for each preexisting node to be connected to any new node at each timestep. 

\subsection{Degree Distribution}
Here we show that regardless of which model we look at, we produce the same master equation governing the evolution of the degree distribution (in particular the master equation for the degree distribution in \cite{ozik2004}).  This master equation is not specific to the spatial structure of the network, or indeed the existence of any spatial structure, but rather relies on the uniform distribution of new links among preexisting nodes.  It appears, in various forms, in networks such as the Deterministic Uniform Random Tree \cite{zhang2008topologies}.

We define $\hat{G}(k,N)$ to be the number of nodes with degree $k$ when we are at time $N$ (i.e. when the system has $N$ nodes).
When a node is added to the network it is initially connected to its $m$ nearest neighbors, so initially (upon creation) $k = m$ for each node, meaning that $\hat{G}(k,N) = 0 \text{ for } k < m$.
At each timestep a node with $k = m$ is added to the network and so connects to $m$ existing nodes.
Since each existing node is equally likely to be chosen to be connected to the new node, there is a $\frac{m}{N}$ probability that any given node with have its degree incremented by 1.
Averaging $\hat{G}(k,N)$ over all possible random node placements we get a master equation for the time evolution of $G(k,N)$, the average of $\hat{G}(k,N)$ over all possible randomly grown networks:

\[G(k,N+1) = G(k,N) - \frac{m}{N}G(k,N) + \frac{m}{N}G(k-1,N) + \delta_{km}\]

where $\delta_{km}$ is the Kronecker delta function.
The first term on the right is the expected number of nodes with degree $k$ at time $N$.
The second term is the expected number of nodes with degree $k$ at time N that get promoted to degree $k+1$.
The third term is the expected number of nodes with degree $k-1$ at time N that get promoted to degree $k$.
The last term on the right is the new node with degree $m$.

It was shown by Ozik et al. \cite{ozik2004} that this master equation leads to an exponentially decaying degree distribution with an asymptotically N invariant form $\bar{H}(k) := \lim_{N \to +\infty} G(k,N)/N$ given by the closed form solution,

\[\bar{H}(k) = \frac{1}{m+1}\left(\frac{m}{m+1}\right)^{k-m}\]

for $k \geq m$ and of course $\bar{H}(k) = 0$ for $k < m$.

The interesting point here is that this exponentially decaying degree distribution comes only from the growing procedure and the uniform probability of attaching new links to existing nodes.  Since, in our models, the uniform probability of attachment to any given node is a result of the uniform spacing of the nodes, conformity to this degree distribution is an important confirmation that our models are, in fact, spatially uniform.
Thus this distribution and the analysis therein holds for all of the generalizations presented in this article. 
So as long as we have a growing network where nodes are added with equal probability of being nearby (in the metric space) any existing node and each new node is connected to its $m$ nearest neighbors, then we will recover the exponential degree distribution given by $\bar{H}(k)$.

%do we want to include figures here to demonstrate our claim? It would be nice to show that theory agrees with simulations for at least a few of the models?

\subsection{Degree-Node Age Distribution}
In nearly all models of growing networks the older nodes experience some preferential attachment, since nodes all have a finite chance each timestep to have new links made, and the older nodes have seen more timesteps and thus have a higher chance of making new links than newer nodes \cite{reallyrandom}.
This inherent preferential attachment provides motivation to explore the relationship between the age of a given node and the degree of that node. 
Based on the work in \cite{reallyrandom} we would expect that older nodes have consistently higher degrees than young nodes, so we seek an expression for the expected degree $k(y,N)$ of a node that has existed for $y$ timesteps given that the network size is $N$ (clearly $y < N$ is required).
Since each node connects to its $m$ nearest neighbors upon creation we start with $k(y,N) = m + (\text{probability of incrementing degree of node})*(\text{timesteps node has seen})$.
But the probability of incrementing the degree of the node is $\frac{m}{N}$, which changes as time progresses, thus we get a sum for our degree-node age distribution,

\[k(y,N) = m + m\sum_{j=0}^{y} \frac{1}{j+N-y}\]

Now this expression works exactly for the expectation value of the degree of a node given its age, but we can simplify the result in the limit of large N by approximating the sum as an integral and then simply integrating to yield a logarithm, so for large $N$ we get the degree-node age distribution,

\[k(y,N) = m + m\log{\frac{N}{N-y}}\]

which we find fits the simulation data well for any value reasonable value of N.
 
Once again we find that since this age distribution was found using only the assumption that each node has an equal chance each timestep to have its degree incremented, this result holds for all of the models discussed here.
This equation represents a specific example of the unavoidable fact that in dynamically growing networks highly connected (and thus older) nodes are preferentially connected to one another discussed in \cite{reallyrandom}.

%once again do we include figures to demonstrate the claim? Which figures to show? They're all pretty much the same since it's the same distribution for every model

\subsection{Betweenness-Degree Relationship}	%this section may not even be necessary...
Betweenness centrality is a common measure of how many shortest paths go through a node; so nodes with higher betweenness appear in more shortest paths across the network than nodes with lower betweenness.
In general we would expect that the nodes with the greatest degree have the highest betweenness since they serve as bridges across the network, meaning many shortest paths between low degree nodes must pass through these high degree nodes.
The existence and subsequent high betweenness of high degree nodes (which can be thought of as hubs in the network) is directly related to the small-world property.
High clustering in networks leads to regions where nodes are highly connected separated from one another.
Short path lengths across the network require these regions to be somehow connected to one another, and this necessitates the existence of hubs to connect the clustered regions.
This means that many low degree nodes aren't necessary for shortest paths across the network since the shortest path will make use of the hubs, leading to higher betweenness for high degree nodes and vice versa.
The variety of models presented here all have the small-world property and share the same degree distribution, and thus we expect they share the same betweenness-degree relationship.

%present figure of betweenness-vs-degree for at least a few of the models?

The main discrepancy between the simulation results and the predictions discussed above appear in the 1 dimensional case.
The high degree nodes in the 1D plum pudding model have much higher betweenness than the high degree nodes in the 1D Thomson model (which is just the OHO model).
The 1D plum pudding model has firm boundary conditions at either edge of the line, and so any shortest path from one side of the line to the other must use ahigh degree node to cross the network.
The OHO model on the other hand has no edge effects and thus the shortest path through the network will go either clockwise or counter-clockwise around the circle, meaning that the shortest paths across the network don't necessarily pass through a select few high degree nodes in the middle (like in the line model).
As we increase the dimension this effect disappears in part because the nodes in higher dimensional spaces are free to move around each other, producing a wider variety of shortest paths than are possible in lower dimensions. 
Thus the discrepancy in betweenness vs. degree in the 1D case can be attributed to the feature that in 1 dimension the nodes behave like a Tonks-Girardeau Gas in that they cannot move past one another.
Although the invariance of the betweenness-degree relationship is not fully established, the higher dimensional models all exhibit the same pattern, indicating that betweenness is not significantly impacted by the space in which the network is embedded as long as the space is sufficiently large to allow nodes to move freely.

\section{CONCLUSION}
%Write the conclusion at the end when we have all of the results and figures and even after we've written the introduction, make sure to keep the conclusion short

\bibliography{grownet}

\end{document}

